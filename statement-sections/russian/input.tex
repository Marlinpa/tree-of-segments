В первой строке задано целое число $n(2 \le n \le 10^5)$ - количество зданий.

Во второй строке через пробел заданы $n$ целых чисел - изначальные высоты зданий $a_i$

$(0 \le a_i \le 10^5$ для $1 \le i \le n - 1$, $a_n = 10^6)$.

В третьей строке задано целое число $q(1 \le q \le 10^5)$ - количество запросов.

Каждая из следующих $q$ строк содержит два целых числа $v$ и $i$ ($1 \le v \le 2$, $1 \le i \le n - 1$) - тип запроса и номер здания соответственно, при $v = 2$ в строке будет находиться еще одно число $x$($1 \le x \le 10^5$) - высота здания после изменения.