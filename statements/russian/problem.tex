\begin{problem}{Бедный Вася и начальство}{стандартный ввод}{стандартный вывод}{3 секунды}{256 мегабайт}

Строитель Вася усиленно трудится на работе, чтоб заработать повышение, ведь дома ему надо кормить пятерых детей! Начальство все никак не хочет продвигать Васю по карьерной лестнице, потому что Вася не может "слепо выполнять все задания, какими бы абсурдными они ни были". Вася и вправду неглуп, чтоб тратить свое время и силы на какие-то странные поручения, но ничего не поделаешь, дети-то кушать хотят. Поэтому Василий решил доказать начальству, что он может все. Тогда начальство дало ему такое задание: 

Есть ряд из $n$ зданий (в них пока еще никто не живет), Васе надо уметь изменять число этажей в каждом здании (сносить или достраивать этажи), а также находить наименьшее здание справа, которое выше данного. Нумерация домов идет слева направо и начинается с 1. Более формально есть $2$ типа запросов:
\begin{enumerate}
  \item На запрос первого типа необходимо вывести значение наименьшего элемента $a_j$ такого, что $a_j > a_i$ и $j > i$;
  \item На запрос второго типа необходимо заменить элемент $a_i$ на $x$.
\end{enumerate}
Помогите же Васе прокормить свою семью!

\InputFile
В первой строке задано целое число $n(2 \le n \le 10^5)$ - количество зданий.

Во второй строке через пробел заданы $n$ целых чисел - изначальные высоты зданий $a_i$

$(0 \le a_i \le 10^5$ для $1 \le i \le n - 1$, $a_n = 10^6)$.

В третьей строке задано целое число $q(1 \le q \le 10^5)$ - количество запросов.

Каждая из следующих $q$ строк содержит два целых числа $v$ и $i$ ($1 \le v \le 2$, $1 \le i \le n - 1$) - тип запроса и номер здания соответственно, при $v = 2$ в строке будет находиться еще одно число $x$($1 \le x \le 10^5$) - высота здания после изменения.

\OutputFile
На каждый запрос первого типа выведите одно число - высоту наименьшего большего дома справа.

\Example

\begin{example}
\exmpfile{example.01}{example.01.a}%
\end{example}

\end{problem}

